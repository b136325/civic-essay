It might be suggested that the stereotype of (not for profit) hackers is of an individual (usually a man) or a group (usually men) of programmers extracting information from (and, thereby, breaking into) a government or corporate computer system. In this gloss, hacking is akin to a high tech heist movie. Another way of understanding hacking might be to consider what it would appear to involve, namely , to hack iteratively at a large problem. And it is this conception of hacking that when applied to social problems gives rise to the term civic hacker. Thus the pre-existing outsider status of hackers is combined with commercial software engineering practices to address civil problems and to do so, for the most part, in a not for profit manner.