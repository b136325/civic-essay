It is suggested that the stereotype of a hacker is of an individual or a group of programmers extracting information (and, thereby, breaking the law) from a government or corporate computer system.
In this conception, hacking has an outlaw status.
Another conception of hacking, however, might consider what it involves, namely, to hack iteratively at a large problem,
and it is this software engineering related conception of hacking that when applied to problems of social justice gives rise to the term civic hacker.
Thus the pre-existing outlaw status of hackers is combined with commercial software engineering practices to address civil problems and to do so, for the most part, in a not for profit manner.