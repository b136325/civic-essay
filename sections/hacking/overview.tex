From a software engineering perspective, hacking might be considered to be an activity that iteratively - and sometimes speedily - simplifies a large problem into smaller units of work.
While that form of problem solving might underlie Marc Zuckerberg's \cite{mark-zuckerberg} conception of \say{moving fast and breaking things},
civic hacking - broadly - might be thought of as the adoption of such an approach (frequently on a voluntary basis) to address issues in civil society.

Furthermore, it is suggested that there are at least two distinct tendencies within the civic hacking movement.
On the one hand, there is the adoption of software engineering processes (both in terms of programming, itself, and as a problem solving structure). 
There are a number of reasons for such an adoption,
including increasingly widespread adoption of technology in civic society,
along with the social influnece of a small number of high profile tech companies, such as Facebook. 
One example of this tendency (at least from an American perspective) might be the national Code for America \cite{code-for-america} organisation.

On the other hand, the second tendency within the civic hacking movement might be considered to be a reaction to the perceived failures (or delays) associated with many large scale government IT projects, such as HealthCare.gov \cite{health-care-gov} (US) or Lorenzo \cite{lorenzo} (UK).
In this tendency,  \say{moving fast and breaking things} dovetails with Ken Thompson's \cite{ken-thompson} Unix design philosophy, which is often summarised as \say{small units loosley coupled}.
Consequently, it is suggested that there is a tendency to valorise the local over the national.

