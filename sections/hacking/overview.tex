From a Software Engineering perspective, hacking frequently means to simplify a large coding problem into smaller units of work through, hopefully, a timely iterative approach to problem solving.
While this form of programming might be considered as being the process underlying Marc Zuckerberg conception of 'moving fast and breaking things',
civic hacking is the adoption of this approach (frequently on a charitable basis) to address issues in civil society.

It is suggested that there are at least two competing tendencies within the civic hacking movement.
On the one hand, there has been the adoption of Software Engineering processes (both in terms of coding, itself, and as a problem solving metaphore) in order to address civic problems. 
This might be a result of the social power - and, indeed, wealth - of a small number of high profile tech companies, such as Facebook, along with the increasingly widespread adoption of technology in civic society.
One example of this tendency (at least from an American perspective) might be the national Code for America organisation.

On the other hand, the second tendency within the civic hacking movement might be considered to be a reaction to the perceived failures (or delays) associated with many large scale government IT projects, such as XXX or XXX.
In this tendency, 'move fast and break things' dovetails with the maxim associated with Unix of 'small units loosley coupled'.
Consequently, it is suggested that there is a tendency to valorise the local over the national.

