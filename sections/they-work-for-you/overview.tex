They Work for You is a civic technology that was founded by Tom Steinberg \cite{tom-steinberg} in 2003.
It provides information about political representatives, such as Members of Parliament (MPs) and Members of the Scottish Parliament (MSPs), across all four legislative assemblies in the UK.
The site is data driven.
That is, it provides an accessible user interface over data drawn, automatically, from secondary sources.
Originally, the data shown within the site was scraped (using a WGET \cite{wget} based script) from the website of Hansard \cite{hansard}, which is the official record of all UK parliamentary debates.
This led to a threat of legal action being taken against They Work for You by the government, because, at the time, no licence existed for the secondary use of Hansard data \cite{they-work-for-you-controversies}.
Whilst no prosecution took place, and whilst Tom Steinberg subsequently co-wrote a governmental review called \emph{The Power of Information} \cite{power-of-information}, They Work for You
no longer scrapes data from the Hansard site, at all.
In fact, since the unveiling of the UK Government's Open Data Licence \cite{open-data-licence} (2010), as overseen by Tim Berners Lee \cite{tim-berners-lee}, They Work for You has drawn data directly from a government Application Programming Interface \cite{api} (or API).