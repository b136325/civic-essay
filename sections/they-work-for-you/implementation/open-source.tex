The third aspect of the implementation of the site that enables it to contribute to civil society is through the provision of open source, freely available code.
To that end, the They Work for You site was redevloped into into a tool called Pombola.
The screen shot within Figure \ref{fig:they-work-for-you-implementation-open-source-pombola} depicts information about Pombola showing that it is currently being used to provide information about the parliaments in both Kenya and South Africa.
The Pombola code \cite{mysociety-github} is freely accessible from the GitHub \cite{github} open source, online code repository, as be seen within Figure \ref{fig:they-work-for-you-implementation-open-source-github}.
It is worth noting that the development of Pombola was financed by the Omidyar Network \cite{omidyar-network}, which is a philanthropic investment organisation, and which was launched by the founder of the aution based e-commerce site eBay \cite{ebay} Pierre Omidyar \cite{pierre-omidyar}.
Furthermore, and highlighting the sometimes complex interplay between civic technologies, the tools that the use, and commercial technologies, GitHub was recenty bought by Microsoft \cite{microsoft-buys-github}.