A second way in which the design of the site engenders accessibility can be seen from the screenshot within Figure \ref{fig:they-work-for-you-implementation-text-only-lynx}.
It depicts a terminal based text only browser, and it is displaying the information from the They Work for You site.
Further information about text only browsing can be found within Appendix B.

Relatively few modern web sites fully support text only browsing, due to the inclusion of \say{Web 2.0} JavaScript \cite{web-2.0-technologies}.
However, text browsers are frequently the basis for text to speech browsing as used by those with disabilities, such as blindness \cite{shelter-guide-web-accessibility}.
As such, the fact that the site can be viewed via a text only browser is another demonstration of how its design engenders accessibility, and, ultimately, contributes to civil society.
