The \emph{pipeline} taxonomy will be used to compare They Work for You with another civic technology, Wikipedia.
Wikipedia was co-founded by Jimmy Wales \cite{jimmy-wales} and Larry Sanger \cite{larry-sanger} in 2001 and has grown since then to become one of the world's largest user maintained repositories of knowledge, with approximately 30 million registered volunteers (or \emph{Wikipedians}) writing and maintaining articles.

From a low level perspective, both They Work for You and Wikipedia are (or were) written in the open source PHP scripting language \cite{php}.
In terms of design and content, Wikipedia (as mentioned) is user maintained by a large group of volunteers.
In contrast, the content within They Work for You is automatically generated from API data feeds.
Moreover, the code handling and displaying the data represented by such feeds would appear to be maintained by a relatively small team of volunteers.
Consequently, it is suggested that the different forms of content generation give rise to differing volunteer groups. Wikipedia volunteers may have a sense of ownerships or involvement with a relatively small part of the overall encyclopedia.  That being said, they are directly concerned with what is displayed on the web. In contrast, the volunteers working for They Work for You may be considered to operate in a more detached manner.

Lastly, whilst both They Work for You and Wikipedia promote the civic power of open knowledge, Wikipedia does so (directly) by enabling end users to become volunteers. 