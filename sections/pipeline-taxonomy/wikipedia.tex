Despite the limitation mentioned, the Pipeline taxonomy will be used to compare They Work for You with another civic technology, which, in this case, will be Wikipedia. Wikipedia was founded by James XXXX in YYYY and has grown since then to become one of the largest user maintained repositories of knowledge, with approximately XXXX volunteers writing and maintaining articles.

From a low level perspective, both They Work for You and Wikipedia are (or were) written in the open source, interpretive PHP scripting language. In terms of design and content, Wikipedia (as mentioned) is user maintained by a large group of volunteers. In contrast, the content within They Work for You is automatically generated from API data feeds. Moreover, the code handling and displaying the data represented by such feeds would appear to be maintained by a relatively small team of volunteers. Consequently, it is suggested that the different forms of content generation give rise to differing volunteer groups. Wikipedia volunteers may have a sense of ownerships or involvement with a relatively small part of the overall encyclopedia.  That being said, they are directly concerned with what is displayed on the web. In contrast, the volunteers working for They Work for You may be considered to operate in a more detached manner.

It is also worth noting that civic technologies may give rise to secondary groups, blurring the distinction between end users and volunteers. Earlier iterations of the They Work for You site contained a ranking of MPs by (in one case) the number of written questions that they had submitted to ministers. As the site become more well known, anecdotal evidence suggested that a small number of MPs were attempting to game the system by asking their advisors to make additional requests. This is an interesting circular example, where (if true) the behaviour of MPs was modified by the presence of the ranking indicator within the site. Please note that They Work for You have since removed such rankings.

Lastly, whilst both They Work for You and Wikipedia promote the civic power of open knowledge, Wikipedia does so (directly) by enabling end users to become volunteers. 