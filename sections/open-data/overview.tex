It is suggested that there are two key tendencies within the broader open data movement.
The first tendency aims to ensure the availability of open and common data licences and standards, as exemplified by the work of Creative Commons \cite{creative-commons} and the World Wide Web Consortium \cite{w3c} (W3C), from a wide range of related organisations.
The second tendency, and with particular regard to politics, attempts to empower civic society with greater levels of knowledge, than would have been the case otherwise, in order to hold governments to account.
A well known example of this tendency is Wikileaks \cite{wikileaks}.  

From an American perspective one of the primary antecedents of the current open data movement is the Freedom of Information Act \cite{foia} (FOIA), which was signed into law by President Lyndon B Johson \cite{lbj} in 1967.
Though more limited in scope than the initial draft legislation championed by Congressman John E Moss \cite{john-e-moss}, and though (at least initially) only concerning the federal government - and with restrictions on classified material - the act nevertheless represented a significant addition to the rights of ordinary American citizens.
Consequently, the introduction of the FOIA might be considered to have increased of the potential of civic society to understand an adminstration and it hold it to account.
To that end, over eight hundred thousands FOIA requests were made in 2017 \cite{foia-requests} (the most recent year for which figures are available).

From a sociological perspective, however, one of the reasons why FOIA requests are made might relate to Anthony Giddens' conception of trust \cite{anthony-giddens-consequences-of-modernity} (as a consequence of modernity).
He suggests that organisational and technological advances mean that many aspects of daily life can not be fully understood (from an individual perspective)
and must, instead, be taken on trust.
A corollary might well be a desire to uncover (as far as possible) the truth beneath such trust.