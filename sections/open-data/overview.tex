It is suggested that there are two key tendencies within the open data movement.
On the one hand, associated organisations aim to ensure the availability of open and common data standards, as exemplified by the work of XXX and YYYY.
On the over hand, and with particular regard to politics, it attempts to empower civil society with a greater level of knowledge, than would have been the case otherwise, in order to hold governments to account.
An obvious example of this tendency is Wikileaks.  

From an American perspective one of the primary antecedents of the current open data movement is the Freedom of Information Act (FOIA), which was signed into law by LBJ in 1963.
Though more limited in scope than the initial draft legislation proposed by Senator Cross, and though (at least initially) only directly concerning the federal government, the act nevertheless represented a significant addition to the rights of ordinary American citizens.
Consequently, the introduction of the FOIA might be considered to have increased of the potential of civic society to hold a government to account.
To that end, between xxx and yyy zxxx citizens made FAOI requests.

From a sociological perspective, moreover, one of the reasons why FOIA requests are made might relate to Giddens conception of trust (within modern and post-modern societies).
He suggests that the organisational and technological advances forming such societies ensure that many aspects of daily life (within them) can not be fully understood (at an individual level).
Consequently, many aspects of daily life be taken on trust.
The corollary might well be a desire to uncover (as far as possible) the truth beneath such trust.
In this conception, trust in governments might be considered to be fragile.
That being said, and as noted by XXXX, negative perceptions of beurocracy and big government would appear to have remained stable since the middle of the 20th century. 

In addition, YYYY noted that some aspectos of open data culture may represent the beurocratisation of civil society and / or the monetisation of data in relation to civil society.