One of the origins of the open data movement can be understood (at least from a sociological perspective) with regard to Anthony Gidden’s conception of trust. Giddens suggested that in modern or post-modern societies, and given the technological advances underlying the development of such societies, knowledge of the aspects of such societies becomes unknowable, producing an almost magical conception of trust. However, Gidden’s suggests that such trust is unstable and that it desires resolution, in terms of knowledge. It is suggested that such desire for understanding underlies at least some aspects of the modern open data movement.

From a historical perspective, however, the movement is often considered to have begun with the Freedom of Information Act (196*) signed into law in America by president LBJ. Although the act was arguably more limited than had been intended, it nevertheless provided a right to information (in this case about federal government).