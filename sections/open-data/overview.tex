It is suggested that there are two key tendencies within the open data movement. On the one hand, individual organizations within it aim to ensure the availability of open and common data standards, as exemplified by the work of XXX and YYYY. On the over hand, and with particular regard to politics, the movement might be considered as aiming to empower civil society with the ability to hold governmental administrations to account. An obvious example of the latter tendency is Wikileaks.  

From an American perspective one of the primary antecedents of the current open data movement is the Freedom of Information Act, which was signed into law by LBJ in 1963. Though more limited in scope than the initial draft legislation proposed by Senator Cross, and though (at least initially) only directly affecting the federal government, the act nevertheless represented a significant addition to the rights of ordinary American citizens. Consequently, the FOIA might be considered to represent an enlargement of the potential of civic society to hold an administration to account, and to that end between xxx and yyy zxxx citizens made FAOI requests.

From a sociological perspective, moreover, one of the reasons for such requests might relate to Giddens conception of trust in modern societies. That is, he suggests that the technological advances which (in part) give rise to such societies nevertheless ensure that many aspects of daily life can not be fully understood and must instead be taken on trust. The corollary of such trust based interactions - or, indeed, relationships - might well be a desire to uncover (as far as possible) the truth underlying the trust. 

That being said, and as noted by XXXX, negative perceptions of beurocracy and big government would appear to have remained stable for around 60 years. 

In addition, YYYY noted that some aspectos of open data culture may represent the beurocratisation of civil society and / or the monetisation of data in relation to civil society.