Micah Sifry’s taxonomy offers a powerful means of \emph{discriminating} between and identifying different types of civic technologies.
However, it is suggested that it does not offer a comprehensive means of \emph{describing} individual projects.
For instance, the taxonomy does not (directly) incorporate the nuanced conception of group membership outlined by Noveck \cite{noveck}.

In addition, it is also worth noting that civic technologies may give rise to \emph{secondary} groups, which blur the distinction between end users and volunteers.
Earlier iterations of the They Work for You site contained a ranking of MPs by (in one case) the number of written questions that the MPs has submitted to ministers.
As the site become more well known, anecdotal evidence suggested that a small number of MPs had attempting to \emph{game} the system by asking their advisors to make additional written requests (using the MPs name), and, thereby, improve their ranking on the site.
This is an interesting circular example, where (if true) the behaviour of MPs was modified by the presence of the ranking indicator within the site.
They Work for You have since removed such rankings.