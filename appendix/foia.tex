From an American perspective one of the primary antecedents of the current open data movement is the Freedom of Information Act \cite{foia} (FOIA), which was signed into law by President Lyndon B Johson \cite{lbj} in 1967.
Though more limited in scope than the initial draft legislation championed by Congressman John E Moss \cite{john-e-moss}, and though (at least initially) only concerning the federal government - and with restrictions on classified material - the act nevertheless represented a significant addition to the rights of ordinary American citizens.
Consequently, the introduction of the FOIA might be considered to have increased of the potential of civic society to understand an adminstration and it hold it to account.
To that end, over eight hundred thousands FOIA requests were made in 2017 \cite{foia-requests} (the most recent year for which figures are available).
